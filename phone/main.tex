\documentclass[journal,12pt,twocolumn]{IEEEtran}
%
\usepackage{setspace}
\usepackage{gensymb}
\usepackage{xcolor}
\usepackage{caption}
%\usepackage{subcaption}
%\doublespacing
\singlespacing
\usepackage{polynom}
%\usepackage{gralphicx}
%\usepackage{amssymb}
%\usepackage{relsize}
\usepackage{mathtools}
%\usepackage{amsthm}
%\interdisplaylinepenalty=2500
%\savesymbol{iint}
%\usepackage{txfonts}
%\restoresymbol{TXF}{iint}
%\usepackage{wasysym}
\usepackage{hyperref}
\usepackage{amsthm}
\usepackage{mathrsfs}
\usepackage{txfonts}
\usepackage{stfloats}
\usepackage{cite}
\usepackage{cases}
\usepackage{subfig}
%\usepackage{xtab}
\usepackage{longtable}
\usepackage{multirow}
%\usepackage{algorithm}
%\usepackage{algpseudocode}
%\usepackage{enumerate}
\usepackage{enumitem}
\usepackage{mathtools}
%\usepackage{iithtlc}
%\usepackage[framemethod=tikz]{mdframed}
\usepackage{listings}
\let\vec\mathbf


%\usepackage{stmaryrd}

\usepackage[cmex10]{amsmath}

%\usepackage{wasysym}
%\newcounter{MYtempeqncnt}
\DeclareMathOperator*{\Res}{Res}
%\renewcommand{\baselinestretch}{2}
\renewcommand\thesection{\arabic{section}}
\renewcommand\thesubsection{\thesection.\arabic{subsection}}
\renewcommand\thesubsubsection{\thesubsection.\arabic{subsubsection}}

\renewcommand\thesectiondis{\arabic{section}}
\renewcommand\thesubsectiondis{\thesectiondis.\arabic{subsection}}
\renewcommand\thesubsubsectiondis{\thesubsectiondis.\arabic{subsubsection}}

%\renewcommand{\labelenumi}{\textbf{\theenumi}}
%\renewcommand{\theenumi}{P.\arabic{enumi}}

% correct bad hyphenation here
\hyphenation{op-tical net-works semi-conduc-tor}

\lstset{
language=Python,
frame=single, 
breaklines=true,
columns=fullflexible
}



\begin{document}
%

\theoremstyle{definition}
\newtheorem{theorem}{Theorem}[section]
\newtheorem{problem}{Problem}
\newtheorem{proposition}{Proposition}[section]
\newtheorem{lemma}{Lemma}[section]
\newtheorem{corollary}[theorem]{Corollary}
\newtheorem{example}{Example}[section]
\newtheorem{definition}{Definition}[section]
%\newtheorem{algorithm}{Algorithm}[section]
%\newtheorem{cor}{Corollary}
\newcommand{\BEQA}{\begin{eqnarray}}
\newcommand{\EEQA}{\end{eqnarray}}
\newcommand{\define}{\stackrel{\triangle}{=}}
\newcommand{\myvec}[1]{\ensuremath{\begin{pmatrix}#1\end{pmatrix}}}
\newcommand{\mydet}[1]{\ensuremath{\begin{vmatrix}#1\end{vmatrix}}}
\bibliographystyle{IEEEtran}
%\bibliographystyle{ieeetr}
\providecommand{\nCr}[2]{\,^{#1}C_{#2}} % nCr
\providecommand{\nPr}[2]{\,^{#1}P_{#2}} % nPr
\providecommand{\mbf}{\mathbf}
\providecommand{\pr}[1]{\ensuremath{\Pr\left(#1\right)}}
\providecommand{\qfunc}[1]{\ensuremath{Q\left(#1\right)}}
\providecommand{\sbrak}[1]{\ensuremath{{}\left[#1\right]}}
\providecommand{\lsbrak}[1]{\ensuremath{{}\left[#1\right.}}
\providecommand{\rsbrak}[1]{\ensuremath{{}\left.#1\right]}}
\providecommand{\brak}[1]{\ensuremath{\left(#1\right)}}
\providecommand{\lbrak}[1]{\ensuremath{\left(#1\right.}}
\providecommand{\rbrak}[1]{\ensuremath{\left.#1\right)}}
\providecommand{\cbrak}[1]{\ensuremath{\left\{#1\right\}}}
\providecommand{\lcbrak}[1]{\ensuremath{\left\{#1\right.}}
\providecommand{\rcbrak}[1]{\ensuremath{\left.#1\right\}}}
\theoremstyle{remark}
\newtheorem{rem}{Remark}
\newcommand{\sgn}{\mathop{\mathrm{sgn}}}
% \providecommand{\abs}[1]{\left\vert#1\right\vert}
% \providecommand{\res}[1]{\Res\displaylimits_{#1}} 
% \providecommand{\norm}[1]{\lVert#1\rVert}
% \providecommand{\mtx}[1]{\mathbf{#1}}
% \providecommand{\mean}[1]{E\left[ #1 \right]}
% \providecommand{\fourier}{\overset{\mathcal{F}}{ \rightleftharpoons}}
% \providecommand{\ztrans}{\overset{\mathcal{Z}}{ \rightleftharpoons}}
% %\providecommand{\hilbert}{\overset{\mathcal{H}}{ \rightleftharpoons}}
% \providecommand{\system}{\overset{\mathcal{H}}{ \longleftrightarrow}}
% 	%\newcommand{\solution}[2]{\textbf{Solution:}{#1}}
\newcommand{\solution}{\noindent \textbf{Solution: }}
\providecommand{\dec}[2]{\ensuremath{\overset{#1}{\underset{#2}{\gtrless}}}}
\numberwithin{equation}{section}
%\numberwithin{equation}{subsection}
%\numberwithin{problem}{subsection}
%\numberwithin{definition}{subsection}
\makeatletter
\@addtoreset{figure}{problem}
\makeatother
\let\StandardTheFigure\thefigure
%\renewcommand{\thefigure}{\theproblem.\arabic{figure}}
\renewcommand{\thefigure}{\theproblem}
%\numberwithin{figure}{subsection}
\def\putbox#1#2#3{\makebox[0in][l]{\makebox[#1][l]{}\raisebox{\baselineskip}[0in][0in]{\raisebox{#2}[0in][0in]{#3}}}}
     \def\rightbox#1{\makebox[0in][r]{#1}}
     \def\centbox#1{\makebox[0in]{#1}}
     \def\topbox#1{\raisebox{-\baselineskip}[0in][0in]{#1}}
     \def\midbox#1{\raisebox{-0.5\baselineskip}[0in][0in]{#1}}
\vspace{3cm}
\title{ 
%\logo{
%}
Pingala Series
%	\logo{Octave for Math Computing }
}
%\title{
%	\logo{Matrix Analysis through Octave}{\begin{center}\includegraphics[scale=.24]{tlc}\end{center}}{}{HAMDSP}
%}
% paper title
% can use linebreaks \\ within to get better formatting as desired
%\title{Matrix Analysis through Octave}
%
%
% author names and IEEE memberships
% note positions of commas and nonbreaking spaces ( ~ ) LaTeX will not break
% a structure at a ~ so this keeps an author's name from being broken across
% two lines.
% use \thanks{} to gain access to the first footnote area
% a separate \thanks must be used for each paragraph as LaTeX2e's \thanks
% was not built to handle multiple paragraphs
%
\author{ Varenya Upadhyaya}

% note the % following the last \IEEEmembership and also \thanks - 
% these prevent an unwanted space from occurring between the last author name
% and the end of the author line. i.e., if you had this:
% 
% \author{....lastname \thanks{...} \thanks{...} }
%                     ^------------^------------^----Do not want these spaces!
%
% a space would be appended to the last name and could cause every name on that
% line to be shifted left slightly. This is one of those "LaTeX things". For
% instance, "\textbf{A} \textbf{B}" will typeset as "A B" not "AB". To get
% "AB" then you have to do: "\textbf{A}\textbf{B}"
% \thanks is no different in this regard, so shield the last } of each \thanks
% that ends a line with a % and do not let a space in before the next \thanks.
% Spaces after \IEEEmembership other than the last one are OK (and needed) as
% you are supposed to have spaces between the names. For what it is worth,
% this is a minor point as most people would not even notice if the said evil
% space somehow managed to creep in.
% The paper headers
%\markboth{Journal of \LaTeX\ Class Files,~Vol.~6, No.~1, January~2007}%
%{Shell \MakeLowercase{\textit{et al.}}: Bare Demo of IEEEtran.cls for Journals}
% The only time the second header will appear is for the odd numbered pages
% after the title page when using the twoside option.
% 
% *** Note that you probably will NOT want to include the author's ***
% *** name in the headers of peer review papers.                   ***
% You can use \ifCLASSOPTIONpeerreview for conditional compilation here if
% you desire.
% If you want to put a publisher's ID mark on the page you can do it like
% this:
%\IEEEpubid{0000--0000/00\$00.00~\copyright~2007 IEEE}
% Remember, if you use this you must call \IEEEpubidadjcol in the second
% column for its text to clear the IEEEpubid mark.
% make the title area
\maketitle
%\newpage
\tableofcontents
%\renewcommand{\thefigure}{\thesection.\theenumi}
%\renewcommand{\thetable}{\thesection.\theenumi}
\renewcommand{\thefigure}{\theenumi}
\renewcommand{\thetable}{\theenumi}
%\renewcommand{\theequation}{\thesection}
\bigskip
\begin{abstract}
This manual provides a simple introduction to Transforms
\end{abstract}
\section{JEE 2019}
Let $\alpha, \beta$ be the roots of the polynomial $x^2-x-1$\\
\begin{align}
	a_n &= \frac{\alpha^{n}-\beta^{n}}{\alpha - \beta}, \quad n \ge 1
	\\
	b_n &= a_{n-1} + a_{n+1}, \quad n \ge 2, \quad b_1 =1
	\label{eq:10-orig-diff}
\end{align}
Verify the following using a python code.
\begin{enumerate}[label=\thesection.\arabic*
,ref=\thesection.\theenumi]
\item 
\begin{align}
	\sum_{k=1}^{n}a_k = a_{n+2}-1, \quad n \ge 1
\end{align}
 \item 
\begin{align}
	\sum_{k=1}^{\infty}\frac{a_k}{10^k} =\frac{10}{89}
\end{align}
 \item 
\begin{align}
	b_n =\alpha^n + \beta^n, \quad n \ge 1 \label{eq:bn}
\end{align}
 \item 
\begin{align}
	\sum_{k=1}^{\infty}\frac{b_k}{10^k} =\frac{8}{89}
\end{align}
\end{enumerate}
\solution The following code verifies the above results:
	    \begin{lstlisting}
wget https://raw.githubusercontent.com/varenya27/EE3900/master/pingala/codes/1.py
\end{lstlisting}
The first three results are true while the last one is false.
\section{Pingala Series}
\begin{enumerate}[label=\thesection.\arabic*,ref=\thesection.\theenumi]
\item The {\em one sided} $Z$-transform of $x(n)$ is defined as 
%\cite{proakis_dsp}
\begin{align}
	X^{+}(z) = \sum_{n = 0}^{\infty}x(n)z^{-n}, \quad z \in \mathbb{C}
\label{eq:one-Z}
\end{align}
	\item The {\em Pingala} series is generated using the difference equation 
\begin{align}
	x(n+2) = x\brak{n+1} + x\brak{n},  \quad x(0) = x(1) = 1, n \ge 0
	\label{eq:10-pingala}
\end{align}
Generate a stem plot for $x(n)$.\\
\solution The following code plots \ref{fig:xn}
	    \begin{lstlisting}
wget https://raw.githubusercontent.com/varenya27/EE3900/master/pingala/codes/2_2.py
\end{lstlisting}
\begin{figure}
    \centering
    \includegraphics[width=\columnwidth]{figures/2_2.png}
    \caption{x(n)}
    \label{fig:xn}
\end{figure}
\item 		Find $X^{+}(z)$.\\
\solution
\begin{align}
    x(n+2)&=x(n+1)+x(n)\\
    \implies \mathcal{Z}^{+}\{x(n+2)\}&=\mathcal{Z}^{+}\{x(n+1)\}+\mathcal{Z}^{+}\{x(n)\}\\
    \implies z^2X^{+}(z)-z^2-z&=zX^{+}(z)-z+X^{+}(z)\\
    \implies X^{+}(z)(z^2-z-1)&=z^2\\
    \implies X^{+}(z) &= \frac{1}{1-z^{-1}-z^{-2}}\\
    &= \frac{1}{\brak{1-\alpha z^{-1}} \brak{1-\beta z^{-1}}}
\end{align}
ROC: $|z|>\frac{1+\sqrt{5}}{2}$

	\item Find $x(n)$.
	\begin{align}
	    X^{+}(z)    &= \frac{1}{\brak{1-\alpha z^{-1}} \brak{1-\beta z^{-1}}}\\
	    &=\brak{\frac{1}{(\alpha-\beta)z^{-1}}}\brak{\frac{1}{1-\alpha z^{-1}} - \frac{1}{1-\beta z^{-1}} }\\
	    &=\frac{1}{z^{-1}(\alpha-\beta)}\brak{\sum_{n=0}^\infty (\alpha z^{-1})^n-(\beta z^{-1})^n }\\
	    &= \sum_{n=0}^\infty \frac{\alpha^n-\beta^n}{\alpha-\beta}z^{-n+1}\\
	    &= \sum_{n=1}^\infty \frac{\alpha^n-\beta^n}{\alpha-\beta}z^{-n+1}\\
	    &= \sum_{n=0}^\infty \frac{\alpha^{n+1}-\beta^{n+1}}{\alpha-\beta}z^{-n}\\
	    \implies x(n) &= \frac{\alpha^{n+1}-\beta^{n+1}}{\alpha-\beta}u(n)\\
	    &=a_{n+1}u(n)\label{eq:xn}
	\end{align}
	\item Sketch 
\begin{align}
	y(n)	 = x\brak{n-1} + x\brak{n+1},  \quad n \ge 0
	\label{eq:10-orig-diff-rev}
\end{align}
 \solution The following code plots \ref{fig:yn}
	    \begin{lstlisting}
wget https://raw.githubusercontent.com/varenya27/EE3900/master/pingala/codes/2_5.py
\end{lstlisting}
\begin{figure}
    \centering
    \includegraphics[width=\columnwidth]{figures/2_5.png}
    \caption{y(n)}
    \label{fig:yn}
\end{figure}
\item Find $Y^{+}(z)$. \\
\solution \begin{align}
    \mathcal{Z}^+\{y(n)\} &= \mathcal{Z}^+\{x(n+1)\}+\mathcal{Z}^+\{x(n-1)\}\\
    Y^+(z) &= zX^+(z)-z+z^{-1}Z^+(z)\\
    &= (z+z^{-1})\frac{1}{1-z^{-1}-z^{-2}}-z\\
    &= \frac{1+2z^{-1}}{1-z^{-1}-z^{-2}}
\end{align}
ROC for the above z transform will again be $|z|>\frac{1+\sqrt{5}}{2}$

\item Find $y(n)$.\\
\solution \begin{align}
    Y^+(z) &= X^+(z) + \frac{2}{\brak{z-\alpha}\brak{z-\beta}}\\
    &= X^+(z) + \frac{2}{\alpha-\beta}\brak{\frac{1}{1-\alpha z^{-1}}+\frac{1}{1-\beta z^{-1}}}\\
    &= X^+(z) +\frac{2}{\alpha-\beta}\brak{\sum_{n=0}^{\infty} (\alpha^n+\beta^n)z^{-n}}\\
    \implies y(n)&= x(n)+2u(n)\frac{\alpha^n+\beta^n}{\alpha-\beta}\\
    &= \frac{\alpha^{n+1}+2\alpha^n - \beta^{n+1}+2\beta^{n}}{\alpha-\beta}u(n)\\
    &= \frac{\alpha^n(\alpha+1)+\alpha^n-\beta^n(\beta+1)+\beta^n}{\alpha-\beta}u(n)\\
    &= \frac{\alpha^{n+2}-\beta^{n+2}-\alpha\beta(\alpha^n+\beta^n)}{\alpha-\beta}u(n)\\
    &= \frac{(\alpha-\beta)(\alpha^{n+1}+\beta^{n+1})}{\alpha-\beta}u(n)\\
    \implies y(n) &= \alpha^{n+1}+\beta^{n+1}\label{eq:yn}
\end{align}

\end{enumerate}

\section{Power of the Z transform}
\begin{enumerate}[label=\thesection.\arabic*,ref=\thesection.\theenumi]
\item Show that 
\begin{align}
	\sum_{k=1}^{n}a_k = 
	\sum_{k=0}^{n-1}x(n) = x(n)*u(n-1)
\end{align}
\solution Using the definition of $x(n)$ from \eqref{eq:xn}\begin{align}
    \sum_{k=1}^na_k &= \sum_{k=0}^{n-1}a_{k+1}\\
    &= \sum_{k=0}^{n-1}x(n)\\
    &= \sum_{k=0}^{n-1}x(k)+x(n)\times0\\
    &= \sum_{k=0}^{n-1}x(k)u(n-1-k)+x(n)\times u(-1)\\
    &= \sum_{k=0}^{n}x(k)u(n-1-k)\\
    &= x(n)*u(n-1) \label{eq:ak}
\end{align}

\item Show that 
\begin{align}
a_{n+2}-1, \quad n \ge 1
\end{align}
can be expressed as 
\begin{align}
	\sbrak{x\brak{n+1}-1}u\brak{n}
\end{align}
\solution The above expression can be written for $n\geq0$ as:
\begin{align}
    a_{n+1}&-1, \quad n\ge0\\
    = x(n)&-1, \quad n\ge0\\
    = (x(n)&-1)u(n)
\end{align}


 \item Show that 
\begin{align}
	\sum_{k=1}^{\infty}\frac{a_k}{10^k}= 
	\frac{1}{10}\sum_{k=0}^{\infty}\frac{x\brak{k}}{10^k} =\frac{1}{10}X^{+}\brak{{10}}
\end{align}
\solution\begin{align}
    	\sum_{k=1}^{\infty}\frac{a_k}{10^k}&=\sum_{k=0}^{\infty}\frac{a_{k+1}}{10^{k+1}}\\
    	&= \sum_{k=0}^{\infty}\frac{x(k)}{10^{k+1}}\\
        &=\frac{1}{10}\sum_{k=0}^{\infty}\frac{x(k)}{10^{k}}\\
        &=\frac{1}{10}\sum_{k=0}^{\infty}x(k)10^{-k}\\
        &= \frac{X^+(10)}{10}\label{eq:az10}
\end{align}

 \item Show that 
\begin{align}
	\alpha^n + \beta^n, \quad n \ge 1
\end{align}
can be expressed as 
\begin{align}
	w(n) =\brak{\alpha^{n+1} + \beta^{n+1}}u(n)\label{eq:wn}
\end{align}
		and find $W(z)$.\\
		\solution Replacing $n$ with $n+1$:
		\begin{align}
		    \alpha^{n+1} + \beta^{n+1}, \quad n \ge 0\\
		    = (\alpha^{n+1} + \beta^{n+1})u(n) = w(n)
		\end{align}
		The z-transform can be computed as follows:
		\begin{align}
		    W(z) &= \sum_{n=-\infty}^{\infty}w(n)z^{-n}\\
		    &=\sum_{n=0}^{\infty}\alpha^{n+1}z^{-n} + \beta^{n+1}z^{-n}\\
		    &=\alpha\sum_{n=0}^{\infty}\alpha^{n}z^{-n} + \beta\sum_{n=0}^{\infty}\beta^{n}z^{-n}\\ 
		    &= \frac{\alpha}{1-\alpha z^{-1}}+\frac{\beta}{1-\beta z^{-1}}\\
		    &= \frac{\alpha+\beta-2\alpha\beta z^{-1}}{1+\alpha\beta z^{-2} -(\alpha+\beta)z^{-1}}\\
		    &= \frac{1+2z^{-1}}{1-z^{-1}-z^{-2}}
		\end{align}
		ROC: $|z|>\frac{1+\sqrt{5}}{2}$
 \item Show that 
\begin{align}
	\sum_{k=1}^{\infty}\frac{b_k}{10^k} =
	\frac{1}{10}\sum_{k=0}^{\infty}\frac{y\brak{k}}{10^k} =\frac{1}{10}Y^{+}\brak{{10}}
\end{align}
\solution 
\begin{align}
        \sum_{k=1}^{\infty}\frac{b_k}{10^k}&=\sum_{k=0}^{\infty}\frac{b_{k+1}}{10^{k+1}}\\
    	&= \sum_{k=0}^{\infty}\frac{x(k)}{10^{k+1}}\\
        &=\frac{1}{10}\sum_{k=0}^{\infty}\frac{y(k)}{10^{k}}\\
        &=\frac{1}{10}\sum_{k=0}^{\infty}y(k)10^{-k}\\
        &= \frac{Y^+(10)}{10}\label{eq:bz10}
\end{align}
\item Solve the JEE 2019 problem.\\
\solution
\begin{enumerate}[a]
    \item From \eqref{eq:ak}
    \begin{align}
        \sum_{k=1}^na_k &= x(n)*u(n-1)
    \end{align}
    Taking the positive z transform on the RHS:
    \begin{align}
        \mathcal{Z}\{x(n)&*u(n-1)\} = X^+(z)z^{-1}\frac{1}{1-z^{-1}}\\
        &= \frac{z^{-1}}{(1-z^{-1}-z^{-2})(1-z^{-1})}\\
        &= z\brak{\frac{1}{1-z^{-1}-z^{-2}} - \frac{1}{1-z^{-1}}}\\
        &= z\sum_{n=0}^\infty (x(n)-1)z^{-n}\\
        &= \sum_{n=0}^\infty (x(n)-1)z^{-n+1}\\
        &= \sum_{n=0}^\infty (x(n+1)-1)z^{-n} \label{eq:akztr}
    \end{align}
    Taking the Inverse Z transform on \eqref{eq:akztr}:
    \begin{align}
        x(n)*u(n-1) &= (x(n+1)-1)u(n)\\
        \implies  \sum_{k=1}^na_k &= a_{n+2}-1,\quad n\geq1
    \end{align}
    
\item  From \eqref{eq:az10}
\begin{align}
    \sum_{k=1}^\infty\frac{a_k}{10^k} &= \frac{X^+(10)}{10}\\
    &= \frac{100}{100-10-1}\times\frac{1}{10}\\
    &= \frac{10}{89}
\end{align}
\item Using \eqref{eq:10-orig-diff}, \eqref{eq:xn}, \eqref{eq:yn} we get:
\begin{align}
    b_n &= a_{n+1}+a_{n-1}\\
    &= x(n)+x(n-2)\\
    &= y(n-1)\\
    &= \alpha^n+\beta^n
\end{align}
\item  From \eqref{eq:bz10}
\begin{align}
    \sum_{k=1}^\infty\frac{b_k}{10^k} &= \frac{Y^+(10)}{10}\\
    &= \frac{100+20}{100-10-1}\times\frac{1}{10}\\
    &= \frac{12}{89} \neq \frac{8}{89}
\end{align}
\end{enumerate}
Thus, options (a),(b) and (c) are correct
\end{enumerate}
\end{document}
