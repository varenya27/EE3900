
\documentclass[journal,12pt,twocolumn]{IEEEtran}
%
\usepackage{setspace}
\usepackage{gensymb}
\usepackage{xcolor}
\usepackage{caption}
\usepackage{tikz}
% \usepackage{decorate}
\usetikzlibrary { decorations.pathmorphing, decorations.pathreplacing, decorations.shapes, }

%\doublespacing
\singlespacing
%\usepackage{graphicx}
%\usepackage{amssymb}
%\usepackage{relsize}
\usepackage[cmex10]{amsmath}
\usepackage{mathtools}
%\usepackage{amsthm}
%\interdisplaylinepenalty=2500
%\savesymbol{iint}
%\usepackage{txfonts}
%\restoresymbol{TXF}{iint}
%\usepackage{wasysym}
\usepackage{hyperref}
\usepackage{amsthm}
\usepackage{mathrsfs}
\usepackage{txfonts}
\usepackage{stfloats}
\usepackage{cite}
\usepackage{cases}
\usepackage{subfig}
%\usepackage{xtab}
\usepackage{longtable}
\usepackage{multirow}
%\usepackage{algorithm}
%\usepackage{algpseudocode}
%\usepackage{enumerate}
\usepackage{enumitem}
\usepackage{mathtools}
%\usepackage{iithtlc}
%\usepackage[framemethod=tikz]{mdframed}
\usepackage{listings}
\let\vec\mathbf

\usepackage{circuitikz}

%\usepackage{stmaryrd}


%\usepackage{wasysym}
%\newcounter{MYtempeqncnt}
\DeclareMathOperator*{\Res}{Res}
%\renewcommand{\baselinestretch}{2}
\renewcommand\thesection{\arabic{section}}
\renewcommand\thesubsection{\thesection.\arabic{subsection}}
\renewcommand\thesubsubsection{\thesubsection.\arabic{subsubsection}}

\renewcommand\thesectiondis{\arabic{section}}
\renewcommand\thesubsectiondis{\thesectiondis.\arabic{subsection}}
\renewcommand\thesubsubsectiondis{\thesubsectiondis.\arabic{subsubsection}}

%\renewcommand{\labelenumi}{\textbf{\theenumi}}
%\renewcommand{\theenumi}{P.\arabic{enumi}}

% correct bad hyphenation here
\hyphenation{op-tical net-works semi-conduc-tor}

\lstset{
language=Python,
frame=single, 
breaklines=true,
columns=fullflexible
}



\begin{document}
\theoremstyle{definition}
\newtheorem{theorem}{Theorem}[section]
\newtheorem{problem}{Problem}
\newtheorem{proposition}{Proposition}[section]
\newtheorem{lemma}{Lemma}[section]
\newtheorem{corollary}[theorem]{Corollary}
\newtheorem{example}{Example}[section]
\newtheorem{definition}{Definition}[section]
\newcommand{\BEQA}{\begin{eqnarray}}
\newcommand{\EEQA}{\end{eqnarray}}
\newcommand{\define}{\stackrel{\triangle}{=}}
\newcommand{\myvec}[1]{\ensuremath{\begin{pmatrix}#1\end{pmatrix}}}
\newcommand{\mydet}[1]{\ensuremath{\begin{vmatrix}#1\end{vmatrix}}}
\bibliographystyle{IEEEtran}
\providecommand{\nCr}[2]{\,^{#1}C_{#2}} % nCr
\providecommand{\nPr}[2]{\,^{#1}P_{#2}} % nPr
\providecommand{\mbf}{\mathbf}
\providecommand{\pr}[1]{\ensuremath{\Pr\left(#1\right)}}
\providecommand{\qfunc}[1]{\ensuremath{Q\left(#1\right)}}
\providecommand{\sbrak}[1]{\ensuremath{{}\left[#1\right]}}
\providecommand{\lsbrak}[1]{\ensuremath{{}\left[#1\right.}}
\providecommand{\rsbrak}[1]{\ensuremath{{}\left.#1\right]}}
\providecommand{\brak}[1]{\ensuremath{\left(#1\right)}}
\providecommand{\lbrak}[1]{\ensuremath{\left(#1\right.}}
\providecommand{\rbrak}[1]{\ensuremath{\left.#1\right)}}
\providecommand{\cbrak}[1]{\ensuremath{\left\{#1\right\}}}
\providecommand{\lcbrak}[1]{\ensuremath{\left\{#1\right.}}
\providecommand{\rcbrak}[1]{\ensuremath{\left.#1\right\}}}
\theoremstyle{remark}
\newtheorem{rem}{Remark}
\newcommand{\sgn}{\mathop{\mathrm{sgn}}}
\providecommand{\fourier}{\overset{\mathcal{F}}{ \rightleftharpoons}}
\providecommand{\ztrans}{\overset{\mathcal{Z}}{ \rightleftharpoons}}
\providecommand{\system}{\overset{\mathcal{H}}{ \longleftrightarrow}}
\newcommand{\solution}{\noindent \textbf{Solution: }}
\providecommand{\dec}[2]{\ensuremath{\overset{#1}{\underset{#2}{\gtrless}}}}
\numberwithin{equation}{section}
\makeatletter

\makeatother

\def\putbox#1#2#3{\makebox[0in][l]{\makebox[#1][l]{}\raisebox{\baselineskip}[0in][0in]{\raisebox{#2}[0in][0in]{#3}}}}
     \def\rightbox#1{\makebox[0in][r]{#1}}
     \def\centbox#1{\makebox[0in]{#1}}
     \def\topbox#1{\raisebox{-\baselineskip}[0in][0in]{#1}}
     \def\midbox#1{\raisebox{-0.5\baselineskip}[0in][0in]{#1}}

\vspace{3cm}

\title{ 
Phone Charger - Lab Report
}
\author{ Varenya Upadhyaya }
\maketitle
\tableofcontents
\bigskip
\begin{abstract}
This report is a summary of the of the procedure and observations found during the making of a 5V DC Phone Charger
\end{abstract}

 \section{Introduction}
 The raw signal from the wall is a 220V 50Hz AC signal. To be able to charge a phone, we need to convert this into a 5V DC signal. 
 This is achieved by using a number of different circuit elements. The procedure and corresponding output wave signals at each step are described below.
 \section{Initial Signal}
 The raw signal is a 220V 50 Hz signal, which is converted into a 12-0-12 step down transformer. The wave signal is shown in \ref{fig:transform}
 \section{Rectifier}
 A four Si diode rectifier is used to rectify the 12V 50Hz AC signal. \ref{fig:rect} shows the output at this stage.
 \section{Filter (Capacitor)}
 The capacitor acts as a low pass filter and only allows the signal with lower frequencies to pass. This creates a noisy DC signal of around 18V shown in \ref{fig:filter}
 \section{Regulator}
This is the final element of the charger and works to convert the noisy 18V signal into a clean 5 DC signal as seen in \ref{fig:final}
The output in the multimeter can be seen in \ref{fig:final_multi}
\section{Figures}
\begin{figure}[h!]
    \centering
    \includegraphics[width=\columnwidth]{figures/transform.jpg}
    \caption{Sine wave at the transformer}
    \label{fig:transform}
\end{figure}

\begin{figure}[h!]
    \centering
    \includegraphics[width=\columnwidth]{figures/rect.jpg}
    \caption{Rectifier output}
    \label{fig:rect}
\end{figure}

\begin{figure}
    \centering
    \includegraphics[width=\columnwidth]{figures/filter.jpg}
    \caption{Output across the low pass filter ie. capacitor}
    \label{fig:filter}
\end{figure}

\begin{figure}
    \centering
    \includegraphics[width = \columnwidth]{figures/final.jpg}
    \caption{Final 5V DC output}
    \label{fig:final}
\end{figure}

\begin{figure}
    \centering
    \includegraphics[width=\columnwidth]{figures/final_multi.jpg}
    \caption{Output shown using the multimeter}
    \label{fig:final_multi}
\end{figure}
\end{document}
